\documentclass[fontsize=12pt]{article}

\usepackage{dawushiyan}

\usepackage{ctex}
\usepackage{float}
\usepackage{amsmath}
\usepackage{fontspec}
\usepackage[margin=6em]{geometry}
\usepackage{scrextend}
\usepackage{anyfontsize}
\usepackage{indentfirst}
\xeCJKsetup{CJKmath=true}

\renewcommand{\d}[1]{\mathrm{d}{#1}}
\newcommand{\diff}[2]{\frac{\d{#1}}{\d{#2}}}
\newcommand{\ddiff}[2]{\dfrac{\d{#1}}{\d{#2}}}
\renewcommand{\phi}{\varphi}
\newcommand{\rel}[1]{\frac{\Delta #1}{#1}}
\newcommand{\drel}[1]{\dfrac{\Delta #1}{#1}}
\newcommand{\ru}[1]{\frac{U_{#1{}0.68}}{#1}}

\title{实验题目\\\Large{实验报告}}
\author{PB20000196 吴天铭}

\begin{document}
  \maketitle

  \begin{labeling}{\textbf{实验题目:}}
    \item[\textbf{实验题目:}]
    \item[\textbf{实验目的:}]
    \item[\textbf{实验原理:}]
    \item[\textbf{实验器材:}]
    \item[\textbf{实验步骤:}]
  \end{labeling}

  \section*{数据处理}

  \section*{思考题}
  \begin{enumerate}
    \item \textbf{题目} \\
      \textbf{答:} 回答。
  \end{enumerate}
\end{document}
